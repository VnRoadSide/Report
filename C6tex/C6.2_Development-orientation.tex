\section{Hướng phát triển}
$\indent$ Trong tương lai, chúng tôi kỳ vọng hệ thống này không chỉ được duy trì mà còn mở rộng các tính năng hiện tại, hướng tới các mục tiêu sau:
    \begin{itemize}
        \item Tối ưu hóa trải nghiệm người dùng bằng cách thiết kế giao diện thân thiện, đẹp mắt, và dễ sử dụng.
        \item Mở rộng phương thức thanh toán, tích hợp thêm các giải pháp thanh toán mới để tăng cường sự tiện lợi và an toàn cho người tiêu dùng và nhà cung cấp.
        \item Cải thiện an ninh mạng, áp dụng các biện pháp bảo mật mạnh mẽ để đảm bảo an toàn cho dữ liệu và thông tin cá nhân của người dùng.
        \item Phát triển chức năng đánh giá và nhận xét sản phẩm, qua đó xây dựng một cộng đồng mua sắm minh bạch và tin cậy.
        \item Khám phá và áp dụng Trí Tuệ Nhân Tạo (AI) và Học Máy (Machine Learning) để cải tiến trải nghiệm mua sắm, cụ thể là:
        \begin{itemize}
            \item Phân loại và lọc sản phẩm tự động, giúp người dùng dễ dàng tìm kiếm và lựa chọn.
            \item Phát triển các mô hình so sánh sản phẩm, đề xuất các lựa chọn thay thế phù hợp dựa trên giá, đánh giá và tính năng.
            \item Triển khai các mô hình gợi ý sản phẩm dựa trên lịch sử mua hàng và hành vi duyệt web của người dùng để cung cấp gợi ý cá nhân hóa.
        \end{itemize}
        \item Tích hợp tính năng gợi ý thông minh vào giao diện người dùng, giúp hiển thị các sản phẩm được đề xuất tùy theo hành vi tìm kiếm của từng người dùng.
    \end{itemize}

Những hướng phát triển này không chỉ nhằm mục đích nâng cao chất lượng trải nghiệm người dùng mà còn đảm bảo hệ thống "Sàn Thương Mại Điện Tử Nông Sản" không chỉ thích ứng linh hoạt với các biến động của thị trường mà còn vượt qua kỳ vọng của người dùng, từng bước hoàn thiện và phát triển.
