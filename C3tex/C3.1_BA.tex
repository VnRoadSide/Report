\section{Phân tích nghiệp vụ chức năng người dùng}
    \subsection{Chủ cửa hàng}
        \subsubsection{Nghiệp vụ quản lý cửa hàng}
        \begin{enumerate}
            \item Tạo cửa hàng: Doanh nghiệp hoặc cá nhân cần tạo cửa hàng trên tài khoản của mình tại trang web thương mại điện tử để bắt đầu kinh doanh.
            \item Tạo sản phẩm bán: Cửa hàng cần tạo sản phẩm và cung cấp đầy đủ thông tin về sản phẩm, bao gồm tên sản phẩm, mô tả sản phẩm, giá cả, hình ảnh, và thông tin vận chuyển.
            \item Thêm số lượng sẵn có: Cửa hàng cần cập nhật số lượng sản phẩm sẵn có để khách hàng có thể biết được sản phẩm còn hàng hay đã hết hàng.
            \item Thêm mô tả cho sản phẩm: Mô tả sản phẩm là một yếu tố quan trọng giúp khách hàng hiểu rõ về sản phẩm. Cửa hàng cần cung cấp mô tả sản phẩm đầy đủ và chi tiết.
            \item Tạo các danh mục sản phẩm: Danh mục sản phẩm giúp khách hàng dễ dàng tìm kiếm sản phẩm. Cửa hàng nên tạo các danh mục sản phẩm rõ ràng và logic để giúp khách hàng tìm kiếm sản phẩm dễ dàng hơn.
            \item Điều chỉnh giá thành của sản phẩm: Giá thành của sản phẩm là một yếu tố quan trọng ảnh hưởng đến quyết định mua hàng của khách hàng. Cửa hàng nên điều chỉnh giá thành của sản phẩm phù hợp với thị trường để tăng doanh số bán hàng.
            \item Tạo các combo sản phẩm: Combo sản phẩm là một cách hiệu quả để tăng giá trị cho khách hàng. Cửa hàng có thể tạo các combo sản phẩm với giá ưu đãi để thu hút khách hàng mua nhiều sản phẩm hơn.
            \item Thêm các chương trình tặng quà khi mua sản phẩm: Các chương trình tăng quà khi mua sản phẩm là một cách hiệu quả để khuyến khích khách hàng mua hàng. Của hàng có thể tạo các chương trình tăng quà khi mua sản phẩm để thu hút khách hàng mới và tăng doanh số bán hàng cho các sản phẩm bán chạy.
        \end{enumerate}
        \subsubsection{Nghiệp vụ quản lý sản phẩm}
        $\indent$ Nghiệp vụ sản phẩm trên sàn thương mại điện tử nông sản bao gồm quá trình quản lý và tiếp thị sản phẩm để thu hút khách hàng, tạo ra doanh số bán hàng, và đảm bảo sự hài lòng của khách hàng. Quản lý sản phẩm trên sàn thương mại điện tử nông sản là một phần quan trọng của hoạt động kinh doanh và đòi hỏi sự chú ý đến chi tiết, hiểu biết về thị trường và nhu cầu của khách hàng, cũng như khả năng tiếp tục tối ưu hóa sản phẩm và chiến dịch tiếp thị.
        \begin{enumerate}
            \item Quản lý Sản Phẩm:
        
        - Thêm, chỉnh sửa, và xóa sản phẩm trên sàn thương mại điện tử.
        
        - Cung cấp thông tin chi tiết về sản phẩm, bao gồm tên, mô tả, hình ảnh, giá cả, số lượng tồn kho, và các thuộc tính kèm theo.
            \item Phân Loại Sản Phẩm:
        Tạo danh mục và phân loại sản phẩm để giúp khách hàng dễ dàng tìm kiếm và duyệt qua các sản phẩm tương tự.
            \item Thông Tin sản phẩm:
        Cung cấp thông tin chi tiết về sản phẩm, ví dụ như trọng lượng, kích thước, xuất xứ, thời hạn sử dụng, và hướng dẫn sử dụng.
            \item Giá và Chiết Khấu:
        - Xác định giá cả cho sản phẩm, bao gồm giá gốc và giá bán.
        
        - Áp dụng chiết khấu hoặc ưu đãi đặc biệt nếu có.
            \item Quản lý Tồn Kho:
        Theo dõi số lượng tồn kho của từng sản phẩm và cập nhật thông tin về số lượng sản phẩm còn lại tức thì.
            \item Quản lý Đánh Giá và Nhận Xét:
        
        - Cho phép khách hàng đánh giá và viết nhận xét về sản phẩm.
        
        - Kiểm duyệt và quản lý đánh giá và nhận xét để đảm bảo tính trung thực và hữu ích.
            \item Quản lý Sản Phẩm Liên Quan:
        Kết nối các sản phẩm tương tự hoặc có liên quan để thúc đẩy giao dịch chéo (cross-selling) và gợi ý sản phẩm.
            \item Tối ưu hóa Sản Phẩm:
        Sản phẩm nên được cập nhật đều đặn để cải thiện tính năng, chất lượng, hoặc cách sử dụng để đáp ứng nhu cầu khách hàng ngày càng biến đổi.
            \item Chăm sóc và Hỗ trợ Sản Phẩm:
        Cung cấp hỗ trợ kỹ thuật, chăm sóc sau bán hàng, và hướng dẫn sử dụng sản phẩm cho khách hàng.
            \item Quản lý Sản Phẩm Bán Chạy và Lưu Lượng:
        Theo dõi sản phẩm nào đang bán chạy và đảm bảo rằng sản phẩm luôn có sẵn trong kho để đáp ứng nhu cầu.
        \end{enumerate}
        
        \subsubsection{Nghiệp vụ bán hàng}
        \begin{enumerate}
            \item Xác nhận đơn hàng:
        
        - Người bán xác nhận đơn hàng và chuẩn bị sản phẩm cho quá trình giao hàng.
        
        - Hệ thống gửi thông báo xác nhận đơn hàng đến người mua.
            \item Giao hàng:
        
        - Người bán chọn phương thức vận chuyển và gửi sản phẩm đến địa chỉ giao hàng được xác định trong đơn hàng.
        
        - Hệ thống cung cấp thông tin vận chuyển để người mua có thể theo dõi quá trình vận chuyển.
        \item  Nhận và kiểm tra sản phẩm:
        
        - Người mua nhận sản phẩm và kiểm tra tính chất và chất lượng của sản phẩm.
        
        - Họ có thể đánh giá sản phẩm sau khi nhận hàng.
            \item Hỗ trợ khách hàng sau bán hàng:
        Sàn thương mại điện tử cung cấp dịch vụ hỗ trợ khách hàng sau bán hàng để giải quyết mọi thắc mắc hoặc khiếu nại của người mua.
            \item Theo dõi lịch sử đơn hàng:
        Hệ thống lưu trữ lịch sử đơn hàng của người mua và người bán, giúp họ theo dõi trạng thái và thông tin chi tiết về đơn hàng trước đây.
            \item Đánh giá và đánh giá sản phẩm:
        Sau khi nhận sản phẩm, người mua có thể đánh giá và viết nhận xét về sản phẩm và dịch vụ của người bán để cung cấp phản hồi cho cộng đồng mua sắm trực tuyến.
        \end{enumerate}
        
        \subsubsection{Nghiệp vụ hỗ trợ khách hàng}
	\begin{enumerate}
            \item Hướng Dẫn Sử Dụng Sản Phẩm:
        Cung cấp hướng dẫn sử dụng chi tiết cho sản phẩm để giúp khách hàng tận dụng tối đa sản phẩm mà họ đã mua.
		\item Xử Lý Phản Hồi Khách Hàng:
        Lắng nghe phản hồi từ khách hàng để cải thiện dịch vụ và sản phẩm.
            \item Trả lời Câu hỏi và Thắc mắc:
        Hỗ trợ người mua trả lời câu hỏi và giải đáp thắc mắc liên quan đến sản phẩm, cách sử dụng, vận chuyển, và chính sách đổi trả.
            \item Phản hồi và Đánh giá:
        Thu thập phản hồi và đánh giá từ người mua để cải thiện chất lượng sản phẩm và dịch vụ.
            \item Chăm sóc Khách hàng Sau Bán hàng:
        Cung cấp dịch vụ chăm sóc khách hàng sau bán hàng để đảm bảo họ hài lòng với sản phẩm và dịch vụ.
	\end{enumerate}
 
        \subsubsection{Quản lý ưu đãi}
        Nghiệp vụ ưu đãi và khuyến mãi trên sàn thương mại điện tử nông sản đóng một vai trò quan trọng trong việc thu hút khách hàng, tạo động lực mua sắm, và tăng doanh số bán hàng. Dưới đây là một số chi tiết về nghiệp vụ này:
        
        \begin{enumerate}
            \item Tạo Chương Trình Khuyến Mãi:
        
        - Xác định mục tiêu và đối tượng của chương trình khuyến mãi (Ví dụ: giảm giá sản phẩm nông sản trong mùa thu).
        
        - Quyết định loại ưu đãi, chẳng hạn như giảm giá tiền mặt, vận chuyển miễn phí, quà tặng kèm, hoặc voucher.
            \item Xác định Điều Kiện Áp Dụng:
        Quy định điều kiện và hạn chế áp dụng cho ưu đãi, bao gồm thời gian, số lượng, loại sản phẩm, và quy định về sử dụng.
            \item Quản Lý Mã Giảm Giá Và Voucher:
        
        - Tạo và quản lý các mã giảm giá và voucher để phân phối cho khách hàng.
        
        - Theo dõi và kiểm soát việc sử dụng mã giảm giá để đảm bảo tính hợp lệ và tuân thủ các quy định.
            \item Quảng Cáo Khuyến Mãi:
        
        - Tạo chiến dịch quảng cáo để thông báo về chương trình khuyến mãi.
        
        - Sử dụng các kênh quảng cáo trực tuyến như quảng cáo trên mạng xã hội, email marketing, và quảng cáo trực tuyến để tiếp cận khách hàng.
            \item Theo Dõi Hiệu Suất:
        Theo dõi hiệu suất chương trình khuyến mãi bằng cách thu thập dữ liệu về số lượng sử dụng, doanh số bán hàng tăng, và ROI (Return on Investment).
            \item Lập Kế Hoạch Và Thời Gian:
        
        - Xác định lịch trình khuyến mãi, bao gồm thời gian bắt đầu và kết thúc của chương trình.
        
        - Đảm bảo rằng sản phẩm và dịch vụ có sẵn trong kho để đáp ứng nhu cầu.
            \item Hỗ trợ Khách Hàng:
        Cung cấp hỗ trợ cho khách hàng liên quan đến chương trình khuyến mãi, bao gồm hướng dẫn sử dụng mã giảm giá và giải quyết khiếu nại.
            
        \end{enumerate}
        
        \subsubsection{Quản lý đánh giá}
        \begin{enumerate}
            \item Tạo khuyến mãi dựa trên đánh giá: Sử dụng đánh giá tích cực để tạo chương trình khuyến mãi và đánh giá sản phẩm nổi bật.
            \item  Theo Dõi Phản Hồi Của Khách Hàng:
        Theo dõi phản hồi từ khách hàng về sản phẩm và dịch vụ để cải thiện chất lượng và trải nghiệm của họ.
            \item Quản Lý Đánh Giá Gắn Kèm Hình Ảnh:
        Cho phép khách hàng đính kèm hình ảnh và video vào đánh giá sản phẩm.
            \item Hiển Thị Đánh Giá Trên Trang Sản Phẩm:
        Đánh giá và đánh giá sản phẩm thường được hiển thị trên trang sản phẩm để giúp người mua xem được đánh giá sản phẩm trước khi mua.
            \item Phản Hồi Đánh Giá:
        Cho phép người bán hoặc quản trị viên trả lời các đánh giá để giải quyết các vấn đề hoặc cung cấp thông tin bổ sung.
            \item Hiển Thị Đánh Giá Gắn Kèm Hình Ảnh:
        Cho phép khách hàng đính kèm hình ảnh hoặc video vào đánh giá sản phẩm để cung cấp thêm thông tin.
            \item Đánh Giá Về Dịch Vụ Giao Hàng:
        Cho phép khách hàng đánh giá dịch vụ giao hàng, bao gồm thời gian giao hàng và tình trạng sản phẩm khi nhận hàng.
            \item Tạo Bài Viết Đánh Giá Tương Tác:
        Cho phép người dùng tương tác với các đánh giá bằng cách bình luận hoặc bỏ phiếu cho những đánh giá có ích.
        \end{enumerate}
        \subsubsection{Nghiệp vụ quản lý hậu thu hoạch}
        Nghiệp vụ quản lý hậu thu hoạch kết hợp với logistics giúp tối ưu hóa các hoạt động sau thu hoạch, bao gồm bảo quản, vận chuyển nông sản. Trên hệ thống, chủ cửa hàng có thể theo dõi chi tiết thông tin sản phẩm như chất lượng, số lượng, thời gian thu hoạch. Họ cũng quản lý các hoạt động logistics như lập lịch trình vận chuyển, theo dõi tình trạng giao hàng theo thời gian thực. Hệ thống còn giúp giám sát tình trạng kho bãi, cảnh báo khi sản phẩm sắp hết hạn sử dụng. 
        \begin{enumerate}
            \item Quản lý thông tin sản phẩm:\\
                - Theo dõi thông tin chi tiết về từng loại sản phẩm như tên, số lượng, chất lượng, thời gian thu hoạch.\\
                - Phân loại, đánh mã định danh cho từng lô hàng.\\
                - Cập nhật tình trạng sản phẩm trong các giai đoạn bảo quản, vận chuyển.\\
            \item Quản lý hoạt động logistics:\\
                - Lập kế hoạch và theo dõi lịch trình vận chuyển, bao gồm phương tiện, tuyến đường, thời gian giao hàng.\\
                - Theo dõi tình trạng vận chuyển theo thời gian thực, như vị trí hiện tại của các lô hàng.\\
            \item Quản lý kho bãi:\\
                - Theo dõi tình trạng kho, như nhiệt độ, độ ẩm, diện tích kho trống.\\
                - Quản lý việc nhập, xuất, tồn kho các lô sản phẩm.\\
                - Cảnh báo khi sản phẩm sắp hết hạn sử dụng hoặc cần phải xử lý.\\
        \end{enumerate}

        \subsubsection{Quản lý đơn đặt trước}
        Đối với một số sản phẩm nông sản, không thể sản xuất hàng loạt và bảo quản lâu dài, cũng như một số sản phẩm chưa đến mùa vụ thì không thể đưa lên sàn bán được. Khi đó, khách hàng có thể tiến hành đặt trước các sản phẩm này và được giao ngay sau khi có hàng. Nghiệp vụ quản lý đơn đặt trước là một trong những hoạt động quan trọng, nó giúp chủ cửa hàng tiếp nhận, xử lý và quản lý các đơn đặt hàng từ khách hàng một cách hiệu quả. Nghiệp vụ này kết hợp chặt chẽ với công tác logistics để đảm bảo hàng hóa được giao đúng thời gian và địa điểm.

        Đa số các hoạt động trong nghiệp vụ này đều giống nghiệp vụ bán hàng, chẳng hạn như tiếp nhận đơn đặt hàng, cập nhật tình trạng đơn, quản lý kho hàng, v.v. đều được thực hiện trên hệ thống của cửa hàng. Tuy nhiên có một số hoạt động cần phải cập nhật thêm các vấn đề liên quan tới tình trạng thu hoạch, sản xuất sản phẩm và thời gian giao hàng.
        \begin{enumerate}
            \item Tiếp nhận đơn đặt trước từ khách hàng: Cửa hàng tiếp nhận đặt trước các sản phẩm nông sản chưa có sẵn, xác nhận các thông tin như tên sản phẩm, số lượng, thời gian giao hàng mong muốn, thông tin liên hệ khách hàng, và xác nhận lại về thời gian giao hàng dự kiến.
            \item Xác nhận khả năng cung ứng: Chủ cửa hàng kiểm tra kế hoạch sản xuất, thu hoạch để đảm bảo có đủ sản phẩm để đáp ứng yêu cầu của khách hàng.Sau đó sẽ xác nhận với khách hàng về khả năng cung ứng.
            \item Theo dõi quá trình sản xuất, thu hoạch: Theo dõi tiến độ sản xuất, thu hoạch để đảm bảo đủ số lượng sản phẩm cho đơn hàng và cập nhật thông tin cho khách hàng về tiến độ và dự kiến thời gian giao hàng.
            \item Quản lý và cập nhật tình trạng đơn đặt trước:\\
                - Xem và quản lý danh sách các đơn đặt hàng từ khách hàng. Xác nhận khả năng cung cấp sản phẩm theo số lượng và thời gian giao hàng yêu cầu.\\
                - Cập nhật tình trạng đơn hàng (đang chờ thu hoạch, đang vận chuyển, đã giao) vào hệ thống.\\
                - Thông báo định kỳ cho khách hàng về tình trạng đơn hàng của họ.
            \item Quản lý giao nhận:Lên lịch giao hàng cho từng đơn đặt hàng, đảm bảo giao đúng thời gian và xử lý các vấn đề phát sinh trong quá trình giao hàng.\\

        \end{enumerate}
    \newpage    
    \subsection{Khách hàng}
        \subsubsection{Mua hàng}
        \begin{enumerate}
            \item Tìm Kiếm Sản Phẩm:
        Người mua truy cập sàn thương mại điện tử và tìm kiếm sản phẩm nông sản theo nhu cầu và sở thích cá nhân.
            \item Duyệt Danh Mục Sản Phẩm:
        Khám phá danh mục sản phẩm, danh mục, và các chế độ lọc để tìm sản phẩm phù hợp.
            \item Xem Thông Tin Sản Phẩm:
        Đọc thông tin chi tiết về sản phẩm, bao gồm hình ảnh, mô tả, giá cả, thuộc tính kỹ thuật, và đánh giá từ người dùng khác.
            \item Thêm Sản Phẩm Vào Giỏ Hàng:
        Chọn sản phẩm và thêm vào giỏ hàng để chuẩn bị cho việc thanh toán.
            \item Kiểm Tra Giỏ Hàng:
        Xem lại danh sách sản phẩm trong giỏ hàng, kiểm tra số lượng và giá cả, và chỉnh sửa nếu cần thiết.
            \item Đặt Hàng:
        Tiến hành đặt hàng bằng cách chọn phương thức thanh toán và giao hàng, nhập thông tin giao hàng, và xác nhận đơn đặt hàng.
            \item Thanh Toán:
        Thực hiện thanh toán bằng cách sử dụng thẻ tín dụng, thẻ ghi nợ, chuyển khoản ngân hàng, hoặc các phương thức thanh toán trực tuyến khác.
            \item Xác Nhận Đơn Hàng:
        Nhận được xác nhận đơn đặt hàng từ sàn thương mại điện tử thông qua email hoặc ứng dụng di động.
            \item Theo Dõi Đơn Hàng:
        Theo dõi trạng thái đơn hàng và thông tin vận chuyển thông qua sàn thương mại điện tử hoặc hệ thống theo dõi đơn hàng trực tuyến.
            \item Nhận Sản Phẩm:
        Nhận sản phẩm nông sản được giao tới địa chỉ giao hàng đã cung cấp và kiểm tra chất lượng và số lượng sản phẩm.
            \item Đánh Giá và Nhận Xét Sản Phẩm:
        Sau khi sử dụng sản phẩm, người mua có thể đánh giá và viết nhận xét về sản phẩm trên sàn thương mại điện tử để chia sẻ thông tin với người dùng khác.
            \item Lưu Trữ Lịch Sử Mua Hàng:
        Sàn thương mại điện tử lưu trữ lịch sử mua hàng của người dùng để thuận tiện cho việc theo dõi và tái đặt hàng trong tương lai.
            \item Hỗ Trợ Khách Hàng:
        Cung cấp hỗ trợ khách hàng liên quan đến quá trình mua hàng, đổi trả, và các câu hỏi khác.
            \item Phí Vận Chuyển và Thuế:
        Trong quá trình đặt hàng, người mua cần kiểm tra các khoản phí vận chuyển và thuế để đảm bảo hiểu rõ giá tổng cộng của đơn hàng.
            \item Phương Thức Thanh Toán:
        Sàn thương mại điện tử nông sản thường cung cấp nhiều phương thức thanh toán khác nhau, bao gồm thẻ tín dụng, thẻ ghi nợ, PayPal, ví điện tử, và chuyển khoản ngân hàng,...
            \item Đánh Giá Tình Trạng Sản Phẩm:
        Trước khi thanh toán, người mua nên kiểm tra kỹ tình trạng sản phẩm để đảm bảo rằng sản phẩm không bị hỏng hoặc thất lạc.
            \item Lựa Chọn Phương Thức Vận Chuyển:
        Người mua có thể chọn phương thức vận chuyển phù hợp với họ, bao gồm giao hàng tận nơi, lấy hàng tại cửa hàng, hoặc vận chuyển ưu đãi.
            \item Theo Dõi Đơn Hàng:
        Cung cấp mã theo dõi để người mua có thể theo dõi trạng thái vận chuyển của đơn hàng và biết khi nào sản phẩm sẽ được giao đến.
            \item An Toàn Thanh Toán:
        Đảm bảo tính an toàn của quá trình thanh toán trực tuyến bằng cách sử dụng các trang web có tích hợp mã hóa và cung cấp hướng dẫn về an toàn thanh toán.
            \item Xử Lý Thanh Toán Hậu Kiểm:
        Người mua nên kiểm tra tài khoản ngân hàng hoặc thẻ tín dụng sau khi thanh toán để xác nhận rằng giao dịch đã được xử lý một cách chính xác.
            \item Hỗ Trợ Trực Tuyến:
        Cung cấp hỗ trợ trực tuyến qua chat trực tiếp, điện thoại hoặc email để giúp người mua giải quyết các vấn đề liên quan đến đơn hàng.
            \item Chương Trình Khách Hàng Thân Thiết:
        Cung cấp các chương trình khách hàng thân thiết và tích điểm để khuyến khích người mua quay lại và mua sắm thường xuyên.
            \item Xác Nhận Giao Dịch:
        Người mua nên nhận được xác nhận giao dịch sau khi hoàn tất việc thanh toán để làm bằng chứng cho đơn hàng của họ.
        \end{enumerate}
        \subsubsection{Đặt trước}
        Như đã nói trên nghiệp vụ Quản lý đơn đặt trước, khách hàng có thể tiến hành đặt trước đối với một số sản phẩm không có sẵn hay các loại nông sản chưa thu hoạch ngay được. Về cơ bản, nghiệp vụ này cũng có các hoạt động giống như nghiệp vụ Mua hàng, chỉ có mốt số điểm khác biệt liên quan đến vấn đề thời gian.
        \begin{enumerate}
            \item Xác định nhu cầu và lựa chọn nhà cung cấp: \\
            - Khách hàng xác định loại sản phẩm nông sản cần mua và thời gian dự kiến nhận hàng.\\
            - Tìm kiếm và lựa chọn các nhà cung cấp uy tín trên hệ thống.
            \item Đặt hàng trước: Khách hàng thực hiện đặt hàng trước sản phẩm mong muốn thông qua tính năng đặt trước trên hệ thống. 
            \item Thỏa thuận các điều khoản: Trên nền tảng, khách hàng và chủ cửa hàng thỏa thuận các điều khoản như giá cả, phương thức thanh toán và thời gian dự kiến giao hàng. Các thông tin này được ghi nhận trong đơn hàng đặt trước.
            \item Theo dõi tiến độ sản xuất và giao hàng: Khách hàng có thể theo dõi tình hình sản xuất, thu hoạch thông qua thông tin cập nhật trên nền tảng.
            \item Nhận hàng và thanh toán: Khách hàng nhận hàng đúng số lượng, chất lượng và thời gian đã thỏa thuận, Kiểm tra sản phẩm và thực hiện thanh toán theo đơn hàng đặt trước.
        \end{enumerate}
        \subsubsection{Giỏ hàng và Đơn hàng}
        \begin{enumerate}
            \item Thêm Sản Phẩm Vào Giỏ Hàng:
        Người mua có thể chọn sản phẩm nông sản và thêm chúng vào giỏ hàng của họ thông qua một nút "Thêm vào giỏ hàng" hoặc biểu tượng tương tự.
        \item Xem Giỏ Hàng:
        Khách hàng có thể xem lại danh sách các sản phẩm đã chọn bằng cách truy cập vào giỏ hàng từ giao diện của họ.
            \item Sửa Sản Phẩm Trong Giỏ Hàng:
        Cho phép người mua thay đổi số lượng sản phẩm hoặc loại bỏ sản phẩm khỏi giỏ hàng nếu họ muốn thay đổi đơn hàng của mình.
            \item Tính Tổng Tiền:
        Hệ thống tự động tính tổng tiền cho toàn bộ đơn hàng bằng cách cộng tất cả các món hàng và tính toán thuế và phí vận chuyển (nếu có).
            \item Sử Dụng Mã Giảm Giá hoặc Voucher:
        Cho phép khách hàng áp dụng mã giảm giá hoặc voucher để nhận được ưu đãi hoặc giảm giá trên tổng giá trị đơn hàng.
            \item Chọn Phương Thức Thanh Toán:
        Khách hàng có thể chọn phương thức thanh toán ưa thích, bao gồm thẻ tín dụng, thẻ ghi nợ, ví điện tử, và chuyển khoản ngân hàng.
            \item Đặt Đơn Hàng:
        Khách hàng cần xác nhận đơn hàng và tiến hành đặt hàng sau khi đã xem xét và chỉnh sửa lại giỏ hàng.
            \item Lưu Giỏ Hàng:
        Cho phép người mua lưu giỏ hàng để hoàn tất đơn hàng sau này hoặc để đánh giá lại danh sách sản phẩm.
            \item Kiểm Tra Số Lượng Còn Trong Kho:
        Hệ thống cần kiểm tra số lượng sản phẩm còn trong kho để đảm bảo rằng sản phẩm có sẵn để bán và không bị thiếu hàng.
            \item Thông Báo Về Sản Phẩm Hết Hạn:
        Thông báo cho người mua nếu có sản phẩm trong giỏ hàng đã hết hàng hoặc không còn sẵn.
            \item Tạo Đơn Hàng Tự Động:
        Sau khi khách hàng đặt hàng thành công, hệ thống tự động tạo một đơn hàng và gửi xác nhận đơn hàng đến khách hàng.
            \item Quản Lý Lịch Sử Giỏ Hàng:
        Lưu trữ lịch sử giỏ hàng của khách hàng để họ có thể xem lại và theo dõi các đơn hàng trước đó.
        \item Hiển Thị Thông Tin Sản Phẩm:
        Cho phép người mua xem thông tin chi tiết về sản phẩm trong giỏ hàng, bao gồm tên sản phẩm, giá, mô tả, và hình ảnh.
            \item Tính Toán Tổng Tiền Chi Phí:
        Tính tổng tiền của các sản phẩm trong giỏ hàng, bao gồm giá sản phẩm, thuế, và phí vận chuyển (nếu có).
            \item Xem Lại Đơn Hàng:
        Khách hàng có thể xem lại đơn hàng để đảm bảo rằng mọi thông tin đều chính xác trước khi hoàn tất thanh toán.
            \item Lưu Trữ Giỏ Hàng Tạm Thời:
        Hệ thống cần lưu trữ giỏ hàng tạm thời để khách hàng có thể tiếp tục mua sắm và quay lại giỏ hàng sau khi đã đăng nhập vào tài khoản của họ.
            \item Xử Lý Mã Giảm Giá hoặc Voucher:
        Cho phép người mua nhập mã giảm giá hoặc voucher để nhận ưu đãi hoặc giảm giá trên tổng giá trị đơn hàng.
            \item Sử Dụng Địa Chỉ Giao Hàng Khác Nhau:
        Cho phép khách hàng sử dụng địa chỉ giao hàng khác với địa chỉ mặc định trong tài khoản của họ nếu cần thiết.
            \item Chọn Phương Thức Thanh Toán Phù Hợp:
        Hệ thống cần hỗ trợ nhiều phương thức thanh toán và hiển thị chúng để người mua có thể chọn phương thức thuận tiện nhất.
            \item Tạo Một Đơn Đặt Hàng Chính Thức:
        Khi khách hàng hoàn tất quá trình xem giỏ hàng và chọn phương thức thanh toán, hệ thống tạo một đơn đặt hàng chính thức.
            \item Xác Nhận Đơn Hàng:
        Hệ thống gửi xác nhận đơn hàng đến khách hàng qua email hoặc thông báo trên trang web để thông báo rằng giao dịch đã thành công.
            \item Theo Dõi Trạng Thái Đơn Hàng:
        Cung cấp cho khách hàng khả năng theo dõi trạng thái đơn hàng, từ khi đặt hàng cho đến khi sản phẩm được giao đến.
            \item Hỗ Trợ Thanh Toán An Toàn:
        Đảm bảo tính an toàn của quá trình thanh toán bằng cách sử dụng kỹ thuật mã hóa và các biện pháp bảo mật khác.
            \item Phát Triển Hệ Thống Giỏ Hàng Thân Thiết:
        Phát triển tính năng giỏ hàng thân thiết để lưu trữ thông tin giỏ hàng của khách hàng và tạo trải nghiệm mua sắm liền mạch trên nhiều thiết bị.
            \item Tích Hợp Hệ Thống Thanh Toán Bên Ngoài:
        Tích hợp với cổng thanh toán bên ngoài như PayPal, Stripe, hoặc các dịch vụ thanh toán trực tuyến khác để xử lý thanh toán.
        \end{enumerate}

        \subsubsection{Nghiệp vụ thanh toán}
        \begin{enumerate}
            \item Lựa Chọn Phương Thức Thanh Toán:
        Cho phép khách hàng chọn phương thức thanh toán, bao gồm thẻ tín dụng, thẻ ghi nợ, ví điện tử, chuyển khoản ngân hàng, hoặc các phương thức thanh toán trực tuyến khác.
            \item Nhập Thông Tin Thanh Toán:
        Khách hàng cần cung cấp thông tin thanh toán, bao gồm số thẻ, ngày hết hạn, mã bảo mật, và thông tin liên quan để thực hiện thanh toán.
            \item Xác Thực Thanh Toán:
        Hệ thống cần thực hiện xác thực thông tin thanh toán để đảm bảo tính xác thực và bảo mật giao dịch.
            \item Ghi Nhận Giao Dịch:
        Ghi nhận thông tin giao dịch, bao gồm số tiền, sản phẩm hoặc dịch vụ mua sắm, và thông tin liên quan về giao dịch.
            \item Phân Loại Giao Dịch:
        Phân loại giao dịch thành các loại khác nhau, ví dụ như thanh toán cho đơn hàng, đặt cọc, hoàn tiền, và các loại giao dịch tài chính khác.
            \item Xử Lý Thanh Toán:
        Gửi thông tin thanh toán đến cơ quan xử lý thanh toán hoặc cổng thanh toán để thực hiện giao dịch tài chính.
            \item Theo Dõi Trạng Thái Thanh Toán:
        Theo dõi trạng thái thanh toán để biết khi nào giao dịch đã hoàn thành và tiền đã được chuyển đến tài khoản người bán.
            \item Xác Nhận Thanh Toán:
        Gửi xác nhận thanh toán đến khách hàng để thông báo rằng giao dịch đã thành công.
            \item Bảo Mật Thanh Toán:
        Đảm bảo tính an toàn và bảo mật thông tin thanh toán bằng cách sử dụng mã hóa và các biện pháp bảo mật khác.
            \item Phí Vận Chuyển và Thuế:
        Tính toán và cộng thêm phí vận chuyển và thuế vào tổng số tiền thanh toán nếu áp dụng.
            \item Lưu Trữ Lịch Sử Thanh Toán:
        Lưu trữ lịch sử các giao dịch thanh toán và cung cấp cho khách hàng truy cập vào lịch sử thanh toán của họ.
            \item Phát Triển Chính Sách Thanh Toán:
        Xây dựng và quản lý chính sách liên quan đến thanh toán, bao gồm chính sách đổi trả và hoàn tiền.
        \end{enumerate}
   